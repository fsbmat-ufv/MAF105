\documentclass[12pt]{beamer}

\usepackage[brazil]{babel}
\usepackage[utf8]{inputenc}
\usepackage[T1]{fontenc}
\usepackage{animate}
\usepackage{amsbsy}
\usepackage{amsfonts}
\usepackage{amsmath}
\usepackage{amssymb}
\usepackage{amsthm}
\usepackage[toc,page,title,titletoc]{appendix}
\usepackage{dsfont}
\usepackage{esvect}
\usepackage[labelfont=bf]{caption}
\usepackage{subcaption}
\usepackage{float}
\usepackage[Glenn]{fncychap}%Sonny %Conny %Lenny %Glenn %Renje %Bjarne %Bjornstrup
\usepackage{graphicx}
\usepackage{indentfirst}%Para indentar os paragrafos automáticamente
\usepackage{lipsum}
\usepackage{longtable}
\usepackage{mathtools}
\usepackage{listings}%Inserir codigo do R no latex
\usepackage{multirow}
\usepackage{multicol}
\usepackage{csquotes}
\usepackage[citestyle=authoryear,maxcitenames=2,terseinits=true,natbib=true, style=abnt]{biblatex}
\addbibresource{Referencias.bib}
\usepackage[figuresright]{rotating}
\usepackage{spalign}
\usepackage{pgfplots}
\pgfplotsset{compat=1.17}
\usepackage{tikz}
\usepackage{color, colortbl}
\usepackage[most]{tcolorbox}
\usepackage{url}
\usepackage{ragged2e}%para justificar o texto dentro de algum ambiente
\definecolor{Gray}{gray}{0.9}
\definecolor{LightCyan}{rgb}{0.88,1,1}

\usepackage[all]{xy}
\usepackage{hyperref,bookmark}
\hypersetup{
  colorlinks=true,
  linkcolor=blue,
  citecolor=red,
  filecolor=blue,
  urlcolor=blue,
}

\usetheme{Madrid}
\usecolortheme[RGB={193,0,0}]{structure}

%\setbeamertemplate{footline}[frame number]
%\setbeamertemplate{footline}[text line]{%
%  \parbox{\linewidth}{\vspace*{-8pt}\hfill\date{}\hfill\insertshortauthor\hfill\insertpagenumber}}
\beamertemplatenavigationsymbolsempty
\renewcommand{\vec}[1]{\mbox{\boldmath$#1$}}
\newtheorem{Teorema}{Teorema}
\newtheorem{Proposicao}{Proposição}
\newtheorem{Definicao}{Definição}
\newtheorem{Corolario}{Corolário}
\newtheorem{Demonstracao}{Demonstração}
\newcommand{\bx}{\ensuremath{\bar{x}}}
\newcommand{\Ho}{\ensuremath{H_{0}}}
\newcommand{\Hi}{\ensuremath{H_{1}}}


\apptocmd{\frame}{}{\justifying}{} % Allow optional arguments after frame.

\title{Estatística Básica}
\author{Prof. Fernando de Souza Bastos\texorpdfstring{\\ fernando.bastos@ufv.br}{}}
\institute{Instituto de Ciências Exatas e Tecnológicas\texorpdfstring{\\ Universidade Federal de Viçosa}{}\texorpdfstring{\\ Campus UFV - Florestal}{}}
\date{}
\newcommand\mytext{Aula 3}
\newcommand\mytextt{Fernando de Souza Bastos}
\newcommand\mytexttt{\url{https://maf105.github.io/}}

\makeatletter
\setbeamertemplate{footline}
{
  \leavevmode%
  \hbox{%
  \begin{beamercolorbox}[wd=.3\paperwidth,ht=2.25ex,dp=1ex,center]{author in head/foot}%
    \usebeamerfont{author in head/foot}\mytext
  \end{beamercolorbox}%
  \begin{beamercolorbox}[wd=.3\paperwidth,ht=2.25ex,dp=1ex,center]{title in head/foot}%
    \usebeamerfont{title in head/foot}\mytextt
  \end{beamercolorbox}%
  \begin{beamercolorbox}[wd=.35\paperwidth,ht=2.25ex,dp=1ex,right]{site in head/foot}%
    \usebeamerfont{site in head/foot}\mytexttt\hspace*{2em}
    \insertframenumber{} / \inserttotalframenumber\hspace*{2ex} 
  \end{beamercolorbox}}%
  \vskip0pt%
}
\makeatother

\providecommand{\arcsin}{} \renewcommand{\arcsin}{\hspace{2pt}\textrm{arcsen}}
\providecommand{\sin}{} \renewcommand{\sin}{\hspace{2pt}\textrm{sen}}
%\newtheorem{Teorema}{Teorema}
%\newtheorem{Proposicao}{Proposição}
%\newtheorem{Definicao}{Definição}
%\newtheorem{Corolario}{Corolário}
%\newtheorem{Demonstracao}{Demonstração}

\titlegraphic{\hspace*{8cm}\href{https://fsbmat-ufv.github.io/}{\includegraphics[width=2cm]{figs/mylogo.png}}
}

% Layout da pagina
\hypersetup{pdfpagelayout=SinglePage}
\begin{document}
%\SweaveOpts{concordance=TRUE}

\frame{\titlepage}

\begin{frame}{}
\frametitle{\bf Sumário}
\tableofcontents
\end{frame}

\section{Tipo de Variáveis}
\begin{frame}{}
\frametitle{Tipo de Variáveis}
\begin{block}{}
\justifying
Conhecer o tipo da variável é importante porque os métodos estatísticos que você pode utilizar em sua análise variam de acordo com o tipo de variável. Existem dois 
tipos principais para variáveis:
\begin{itemize}
\item {\bf Variáveis categóricas} (ou {\bf Variáveis qualitativas}) apresentam valores que podem somente ser posicionados em categorias tais como sim e não. \pause
\item {\bf Variáveis numéricas} (ou {\bf Variáveis quantitativas}) apresentam valores que representam quantidades.
\end{itemize}
\end{block}
\end{frame}

\begin{frame}{}
\frametitle{Tipo de Variáveis}
\begin{block}{}
\justifying
Dentre as variáveis qualitativas, ainda podemos fazer uma distinção entre dois tipos: \textbf{variável qualitativa nominal}, para a qual não existe nenhuma ordenação nas
possíveis realizações, e \textbf{variável qualitativa ordinal}, para a qual existe uma ordem nos seus resultados.
\end{block}
\end{frame}

\begin{frame}{}
\frametitle{}
\begin{block}{}
\justifying
Variáveis numéricas podem ser, ainda, identificadas como \textbf{variáveis discretas} ou \textbf{variáveis contínuas}.
\end{block}
\pause
\begin{block}{}
\justifying
\textbf{Variáveis discretas} apresentam valores numéricos que surgem a partir de um processo de contagem. ``A quantidade de canais de TV a Cabo Premium que você 
assina'' é um exemplo de uma variável numérica discreta, uma vez que a resposta corresponde a um entre uma quantidade finita de números inteiros.
\end{block}
\end{frame}

\begin{frame}{}
\frametitle{}
\begin{block}{}
\justifying
Variáveis contínuas produzem respostas numéricas que surgem a partir de um processo de medição. O tempo que você espera pelo atendimento de um caixa no banco 
é um exemplo de variável numérica contínua, uma vez que a resposta pode assumir qualquer valor dentro dos limites de um continuum, ou de um intervalo.
\end{block}
\end{frame}

\begin{frame}{}
\frametitle{}
\begin{block}{}
\justifying
Em uma primeira análise, identificar o tipo da variável pode parecer fácil, embora algumas variáveis que você poderia desejar estudar possam ser categóricas ou numéricas, 
dependendo do modo como você as define. Por exemplo, ``idade'' aparentaria ser uma variável numérica evidente, mas o que acontece se você estiver interessado em 
comparar os hábitos de compra de crianças, adolescentes, pessoas de meia-idade e pessoas com idade para aposentadoria? Nesse caso, definir ``idade'' como uma 
variável categórica faria mais sentido. 
\end{block}
\end{frame}

\begin{frame}{}
\frametitle{Introdução}
\begin{block}{}
\justifying
\begin{figure}[H]
    \centering
    \caption{Classificação de uma variável}
    \begin{tikzpicture}
    \node at (-8,0) {Variável};
    \draw[->](-7,0)--(-5,1);
    \draw[->](-7,0)--(-5,-1);
    \node at (-4,1) {Qualitativa};
    \node at (-3.9,-1) {Quantitativa};
    \draw[->](-3,1)--(0,1.5);
    \draw[->](-3,1)--(0,0.5);
    \node at (1,1.5) {Nominal};
    \node at (1,0.5) {Ordinal};
    \draw[->](-2.8,-1)--(0,-0.5);
    \draw[->](-2.8,-1)--(0,-1.5);
    \node at (1,-0.5) {Discreta};
    \node at (1,-1.5) {Contínua};
    \end{tikzpicture}
    \subcaption*{\textbf{Fonte:} \cite{morettin2017estatistica}}
    \label{Fig2_ex}
  \end{figure}
\end{block}
\end{frame}

\begin{frame}{}
\frametitle{}
\begin{block}{}
\justifying
Apesar de não indicado na maioria das vezes, podemos transformar variáveis quantitativas em qualitativas. Considere, por exemplo, que a estatura de um homem adulto varie entre $1,40$ e $2,20$ m. Quando coletarmos dados dos elementos da população, podemos encontrar absolutamente qualquer valor dentro deste intervalo. Caso não queira usar os dados em seu estado bruto, quantitativo, podemos transformar a variável em qualitativa categorizando-a em grupos, por exemplo, baixos (menor que $1,60$ m), médios ($1,60-1,80$ m) e altos (maior que $1,80$m). 
\end{block}
\pause
\begin{block}{}
Notem, no entanto, que perdemos muita informação nesse tipo de transformação!
\end{block}
\end{frame}

\begin{frame}{}
\frametitle{}
\begin{block}{}
\justifying
Podemos também atribuir valores numéricos às várias qualidades ou atributos (ou, ainda, classes) de uma variável qualitativa e depois proceder-se à análise como se esta fosse quantitativa, desde que o procedimento seja passível de interpretação. 
\end{block}
\pause
\begin{block}{}
Por exemplo, variáveis dicotômicas para as quais só podem ocorrer duas realizações, usualmente chamadas sucesso e fracasso. Podemos nesse caso associar o fracasso ao valor zero e o sucesso ao valor 1.
\end{block}
\end{frame}

\begin{frame}{}
\frametitle{}
\begin{block}{}
\justifying
Para cada tipo de variável existem técnicas apropriadas para resumir as informações e o maior interesse do pesquisador é conhecer o comportamento da variável, analisando a ocorrência de suas possíveis realizações. Vejamos uma maneira de se dispor um conjunto de realizações, para se ter uma ideia global sobre elas, ou seja, de sua distribuição.
\end{block}
\end{frame}

\section{Distribuições de Frequências}
\begin{frame}{}
\frametitle{Banco de Dados}
\begin{block}{}
\justifying
Um pesquisador está interessado em fazer um levantamento sobre alguns aspectos socioeconômicos dos empregados da seção de orçamentos da Companhia MB. Usando informações obtidas do departamento pessoal, ele elaborou a Tabela \href{https://raw.githack.com/maf105/maf105.github.io/master/Aulas_MAF105/Aula1/CompanhiaMB.html}{CompanhiaMB}.
%\begin{table}[H]
%%\caption{My caption}
%\label{tab1}
%\begin{tabular}{c|c}
%\hline
%Variável             &Representação\\
%\hline
%Estado Civil         &$X$\\
%Grau de Instrução    &$Y$\\
%Número de filhos     &$Z$\\
%Salário              &$S$\\
%Idade                &$U$\\
%Região de Procedência&$V$\\
%\hline
%\end{tabular}
%\end{table}
\end{block}
\end{frame}


%\begin{frame}{}
%\frametitle{Distribuições de Frequências}
%\begin{block}{}
%\justifying
%Quando se estuda uma variável, o maior %interesse do pesquisador é conhecer o %comportamento dessa variável, analisando a %ocorrência de suas possíveis realizações. 
%Vejamos uma maneira de se dispor um conjunto %de realizações, para se ter uma idéia global %sobre elas, ou seja, de sua distribuição.
%\end{block}
%\end{frame}

\begin{frame}{}
\frametitle{Distribuições de Frequências}
\vspace{-0.5cm}
\begin{block}{}
\justifying
\begin{table}[H]
\caption{Frequências e porcentagens dos 36 empregados da seção de orçamentos da Companhia MB segundo o grau de instrução.}
\label{tab2}
\begin{tabular}{c|c|c|c}
\hline
Grau de   &Frequência&Proporção&Porcentagem\\
instrução &$n_{i}$   &$f_{i}$  &$100f_{i}$ \\
\hline
Fundamental&12       &0,3333   &33,33      \\
Médio      &18       &0,5000   &50,00      \\
Superior   & 6       &0,1667   &16,67      \\
\hline
Total      &36       &1,0000   &100,00     \\
\hline
\end{tabular}
\subcaption*{\textbf{Fonte:} \cite{morettin2017estatistica}}
\end{table}
\end{block}
\pause
\vspace{-0.3cm}
\begin{block}{}
\justifying
Observando os resultados da segunda coluna, vê-se que dos 36 empregados da companhia,
12 têm o ensino fundamental, 18 o ensino médio e 6 possuem curso superior.
\end{block}
\end{frame}

\begin{frame}{}
\frametitle{}
\begin{block}{}
\justifying
Uma medida bastante útil na interpretação de tabelas de frequências é a proporção de
cada realização em relação ao total, pois podem ser utilizadas quando se quer comparar
resultados de duas pesquisas distintas.
\end{block}
\pause
\begin{block}{}
\justifying
Por exemplo, suponhamos que se queira comparar a variável grau de instrução para empregados da seção de orçamentos com a mesma variável para todos os empregados da Companhia MB. Digamos que a empresa tenha 2.000 empregados e que a distribuição de frequências seja a da próxima Tabela.
\end{block}
\end{frame}

\begin{frame}{}
\frametitle{}
\begin{block}{}
\justifying
\begin{table}[H]
\caption{frequências e porcentagens dos dos 2.000 empregados da Companhia MB segundo o grau de instrução.}
\label{tab3}
\begin{tabular}{c|c|c|c}
\hline
Grau de   &Frequência&Proporção&Porcentagem\\
instrução &$n_{i}$   &$f_{i}$  &$100f_{i}$ \\
\hline
Fundamental&650      &0.325    &32.50      \\
Médio      &1.020    &0.51     &51.00      \\
Superior   & 330     &0.165    &16.50      \\
\hline
Total      &2.000    &1,0000   &100.00     \\
\hline
\end{tabular}
\subcaption*{\textbf{Fonte:} \cite{morettin2017estatistica}}
\end{table}
\end{block}
\begin{block}{}
Não podemos comparar diretamente as colunas das frequências das Tabelas \ref{tab2} e \ref{tab3}, pois os totais de empregados são diferentes nos dois casos. Mas 
as colunas das porcentagens são comparáveis, pois reduzimos as frequências a um mesmo total (no caso 100).
\end{block}
\end{frame}

\begin{frame}{}
\frametitle{}
\begin{block}{}
\justifying
A construção de tabelas de frequências para variáveis contínuas necessita de certo
cuidado. Por exemplo, a construção da tabela de frequências para a variável salário,
usando o mesmo procedimento acima, \textbf{não resumirá as 36 observações} num grupo
menor, pois não existem observações iguais. A solução empregada é agrupar os dados
por faixas de salário.
\end{block}
\end{frame}

\begin{frame}{}
\frametitle{}
\begin{block}{}
\justifying
\begin{table}[H]
\caption{Frequências e porcentagens dos dos 2.000 empregados da seção de orçamentos da Companhia MB por faixa de salário.}
\label{tab4}
\begin{tabular}{c|c|c}
\hline
Classe de   &Frequência&Porcentagem\\
Salários    &$n_{i}$   &$100f_{i}$ \\
\hline
4,00|-8,00  &10        &27,78      \\
8,00|-12,00 &12        &33,33      \\
12,00|-16,00&8         &22,22      \\
16,00|-20,00&5         &13,89      \\
20,00|-24,00&1         & 2,78      \\
\hline
Total       &36        &100,00     \\
\hline
\end{tabular}
\end{table}
\end{block}
\end{frame}

\begin{frame}{}
\frametitle{}
\begin{block}{}
\justifying
Procedendo-se desse modo, ao resumir os dados referentes a uma variável contínua,
perde-se alguma informação. Por exemplo, não sabemos quais são os oito salários da
classe de 12 a 16, a não ser que investiguemos a tabela original. Sem perda de muita precisão, poderíamos supor que todos os oito salários daquela classe fossem iguais ao ponto médio da referida classe, isto é, 14.
\end{block}
\end{frame}

\begin{frame}{}
\frametitle{}
\begin{block}{}
\justifying
Note que estamos usando a notação $a|-b$ para o intervalo de números contendo o extremo $a$ mas não contendo o extremo $b.$ Podemos também usar a notação $[a, b)$ para designar o mesmo intervalo $a|-b$. 
\end{block}
\end{frame}

\begin{frame}{}
\frametitle{}
\begin{block}{}
\justifying
A escolha dos intervalos é arbitrária e a familiaridade do pesquisador com os dados é
que lhe indicará quantas e quais classes (intervalos) devem ser usadas.
\begin{itemize}
\item Número pequeno de classes $\rightarrow$ perda de informação;\pause
\item Número grande de classes $\rightarrow$ perda da visão geral dos dados
como um conjunto;\pause
\item A sugestão é usar de 5 a 15 classes com a mesma amplitude;
\end{itemize}
\end{block}
\end{frame}

\begin{frame}{}
\frametitle{}
\begin{block}{}
\justifying
Para construir uma distribuição de frequências separando por classes uma determinada variável podemos utilizar:
\begin{itemize}
\item número de classes($n_{c}$)$\approx \sqrt{n}$ ou usamos a regra de Sturges $n_{c}=\ln{(n)};$\pause
\item Amplitude da classe $(c)$: $$c=\dfrac{AT}{n_{c}-1}$$ 
\end{itemize}
em que $AT=\textrm{Maior valor} - \textrm{Menor valor}.$
\end{block}
\end{frame}

\begin{frame}{}
\begin{block}{}
Considere
\begin{itemize}
    \item $x_{(1)}$ como o menor valor do conjunto de dados;
    \item $x_{(n)}$ como o maior valor do conjunto de dados; 
\end{itemize}
\end{block}
\pause
\begin{block}{}
\begin{itemize}
\item Defina $Li_{1}=x_{(1)}-\dfrac{c}{2},$ $Ls_{k}=Li_{k}+c,\ 1\leq k \leq n_{c}$ e
$Li_{k}=Ls_{(k-1)},2 \leq k \leq n_{c}.$ Em que $Li_{k}$ representa o limite inferior da classe $k$ e $Ls_{k}$ representa o limite superior da classe $k.$
\item A frequência absoluta $(f_{a})$ de uma classe $k$ é encontrada contabilizando os valores $x_{1},x_{2},\cdots,x_{n}$ que pertencem ao intervalo $[Li_{k},Ls_{k}).$
\end{itemize}
\end{block}
\end{frame}

\begin{frame}{}
\frametitle{}
\begin{block}{}
\justifying
Na página da disciplina há um material sobre distribuição de frequências com um exemplo. Para acessá-lo, \href{https://rawgit.com/maf105/maf105.github.io/master/Materiais/Dist_freq.pdf}{clique aqui}!
\end{block}
\end{frame}

\begin{frame}%[allowframebreaks]
\frametitle{\bf Referências}
\printbibliography
\end{frame}


\end{document}

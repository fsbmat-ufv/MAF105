\documentclass[14pt,aspectratio=1610]{beamer}

\usepackage[brazil]{babel}
\usepackage[utf8]{inputenc}
%\UseRawInputEncoding
\usepackage[T1]{fontenc}
\usepackage{Sweave}
\usepackage{animate}
\usepackage{amsbsy}
\usepackage{amsfonts}
\usepackage{amsmath}
\usepackage{amssymb}
\usepackage{amsthm}
\usepackage[toc,page,title,titletoc]{appendix}
%\usepackage[fixlanguage]{babelbib}
%\usepackage[pdftex]{color}
\usepackage{dsfont}
\usepackage{esvect}
\usepackage[labelfont=bf]{caption}
\usepackage{subcaption}
\usepackage{float}
\usepackage[Glenn]{fncychap}%Sonny %Conny %Lenny %Glenn %Renje %Bjarne %Bjornstrup
%\usepackage{geometry, calc, color, setspace}%
%\geometry{a4paper, headsep=1.0cm, footskip=1cm, lmargin=3cm, rmargin=2cm, tmargin=3cm, bmargin=2cm}
\usepackage{graphicx}
\usepackage{indentfirst}%Para indentar os parágrafos automáticamente
\usepackage{lipsum}
\usepackage{longtable}
\usepackage{mathtools}
\usepackage{listings}%Inserir codigo do R no latex
\usepackage{multirow}
\usepackage{multicol}
\usepackage{natbib}
\setcitestyle{authoryear,open={(},close={)}} %Citation-related commands
\bibliographystyle{abbrvnat}
%\usepackage{csquotes}
%\usepackage[natbib=true,style=abnt, sorting=none]{biblatex}
%\addbibresource{bibliografia.bib}
\usepackage[figuresright]{rotating}
\usepackage{spalign}
%\usepackage{pgfpages}
\usepackage{pgfplots}
\pgfplotsset{compat=1.18}
\usepackage{tikz}
\usepackage{color, colortbl}
\usepackage{ragged2e}%para justificar o texto dentro de algum ambiente
\definecolor{Gray}{gray}{0.9}
\definecolor{LightCyan}{rgb}{0.88,1,1}
%\usepackage{grffile}

\usepackage[all]{xy}



\usetheme{Madrid}
\usecolortheme[RGB={193,0,0}]{structure}

%\setbeamertemplate{footline}[frame number]
%\setbeamertemplate{footline}[text line]{%
%  \parbox{\linewidth}{\vspace*{-8pt}\hfill\date{}\hfill\insertshortauthor\hfill\insertpagenumber}}
\beamertemplatenavigationsymbolsempty
\renewcommand{\vec}[1]{\mbox{\boldmath$#1$}}
\newtheorem{Teorema}{Teorema}
\newtheorem{Proposicao}{Proposição}
\newtheorem{Definicao}{Definição}
\newtheorem{Corolario}{Corolário}
\newtheorem{Demonstracao}{Demonstração}
\newcommand{\bx}{\ensuremath{\bar{x}}}
\newcommand{\Ho}{\ensuremath{H_{0}}}
\newcommand{\Hi}{\ensuremath{H_{1}}}


\apptocmd{\frame}{}{\justifying}{} % Allow optional arguments after frame.

\title{Estatística Básica}
\author{Prof. Fernando de Souza Bastos\texorpdfstring{\\ fernando.bastos@ufv.br}{}}
\institute{Instituto de Ciências Exatas e Tecnológicas\texorpdfstring{\\ Universidade Federal de Viçosa}{}\texorpdfstring{\\ Campus UFV - Florestal}{}}
\date{}
\newcommand\mytext{Aula 6}
\newcommand\mytextt{Fernando de Souza Bastos}
\newcommand\mytexttt{\url{https://maf105.github.io/}}

\makeatletter
\setbeamertemplate{footline}
{
  \leavevmode%
  \hbox{%
  \begin{beamercolorbox}[wd=.3\paperwidth,ht=2.25ex,dp=1ex,center]{author in head/foot}%
    \usebeamerfont{author in head/foot}\mytext
  \end{beamercolorbox}%
  \begin{beamercolorbox}[wd=.3\paperwidth,ht=2.25ex,dp=1ex,center]{title in head/foot}%
    \usebeamerfont{title in head/foot}\mytextt
  \end{beamercolorbox}%
  \begin{beamercolorbox}[wd=.35\paperwidth,ht=2.25ex,dp=1ex,right]{site in head/foot}%
    \usebeamerfont{site in head/foot}\mytexttt\hspace*{2em}
    \insertframenumber{} / \inserttotalframenumber\hspace*{2ex} 
  \end{beamercolorbox}}%
  \vskip0pt%
}
\makeatother

\providecommand{\arcsin}{} \renewcommand{\arcsin}{\hspace{2pt}\textrm{arcsen}}
\providecommand{\sin}{} \renewcommand{\sin}{\hspace{2pt}\textrm{sen}}
%\newtheorem{Teorema}{Teorema}
%\newtheorem{Proposicao}{Proposição}
%\newtheorem{Definicao}{Definição}
%\newtheorem{Corolario}{Corolário}
%\newtheorem{Demonstracao}{Demonstração}

\titlegraphic{\hspace*{8cm}\href{https://fsbmat-ufv.github.io/}{\includegraphics[width=2cm]{figs/mylogo.png}}
}


\usepackage{hyperref,bookmark}
\hypersetup{
  colorlinks=true,
  linkcolor=blue,
  citecolor=red,
  filecolor=blue,
  urlcolor=blue,
}

% Layout da pagina
\hypersetup{pdfpagelayout=SinglePage}
\begin{document}
\input{Aula6-concordance}

\frame{\titlepage}

\begin{frame}{}
\frametitle{\bf Sumário}
\tableofcontents
\end{frame}

\section{Medidas de Posição}
\begin{frame}{}
\frametitle{}
\begin{block}{}
\justifying
Vimos que o resumo de dados por meio de tabelas de frequências e ramo-e-folhas fornece
muito mais informações sobre o comportamento de uma variável do que a própria tabela
original de dados. 
\end{block}
\pause
\begin{block}{}
\justifying
Muitas vezes, queremos resumir ainda mais estes dados, apresentando um ou alguns valores que sejam representativos da série toda. \textbf{Quando usamos um só 
valor, obtemos uma redução drástica dos dados.} Usualmente, emprega-se uma das seguintes medidas de posição (ou localização) central: \bf{média}, \bf{mediana} ou \bf{moda}.
\end{block}
\end{frame}

\begin{frame}{}
\frametitle{}
\begin{block}{}
\justifying
A moda é definida como a realização mais frequente do conjunto de valores observados. Por exemplo, considere a variável $Z,$ número de filhos de cada funcionário 
casado, resumida na Tabela de dados da companhia MB. Vemos que a moda é 2, correspondente à realização com maior frequência, 7. Em alguns casos, pode haver 
mais de uma moda, ou seja, a distribuição dos valores pode ser bimodal, trimodal etc.
\end{block}
\end{frame}

\begin{frame}{}
\frametitle{}
\begin{block}{}
\justifying
A mediana é a realização que ocupa a posição central da série de observações, quando
estão ordenadas em ordem crescente. Assim, se as cinco observações de uma variável forem $3, 4, 7, 8$ e $8,$ a mediana é o valor $7,$ correspondendo à terceira 
observação. Quando o número de observações for par, usa-se como mediana a média aritmética das duas observações centrais. Acrescentando-se o valor $9$ à série acima, a mediana será $(7 + 8)/2 = 7,5.$
\end{block}
\end{frame}

\begin{frame}{}
\frametitle{}
\begin{block}{}
\justifying
Finalmente, a média aritmética, conceito familiar ao leitor, é a soma das observações dividida pelo número de observações.
\end{block}
\end{frame}

%\begin{frame}{}
%\frametitle{}
%\begin{block}{}
%\justifying
%Neste exemplo, as três medidas têm valores próximos e qualquer uma delas pode ser
%usada como representativa da série toda. A média aritmética é, talvez, a medida mais usada.
%Contudo, ela pode conduzir a erros de interpretação. Em muitas situações, a mediana é uma
%medida mais adequada. Veremos algumas situações que ilustram tal afirmação.
%\end{block}
%\end{frame}

\begin{frame}{}
\frametitle{}
\begin{block}{}
\justifying
Se $x_{1},\cdots,x_{n}$ são os $n$ valores (distintos ou não) da variável $X,$ a média aritmética, ou simplesmente média, de $X$ pode ser escrita como:
\begin{equation}
\bx=\dfrac{x_{1}+\cdots+x_{n}}{n}=\dfrac{\displaystyle \sum_{i=1}^{n}{x_{i}}}{n}
\end{equation}
Agora, se tivermos $n$ observações da variável $X,$ das quais $n_{1}$ são iguais a $x_{1},$ $n_{2}$ são iguais
a $x_{2},$ $n_{k}$ iguais a $x_{k},$ então a média de $X$ pode ser escrita como:
\begin{equation}
\bx=\dfrac{n_{1}x_{1}+\cdots+n_{k}x_{k}}{n_{1}+\cdots+n_{k}}=\dfrac{\displaystyle \sum_{i=1}^{k}n_{i}x_{i}}{n}
\end{equation}
\end{block}
\end{frame}

\begin{frame}{}
\frametitle{}
\begin{block}{}
\justifying
Consideremos, agora, as observações ordenadas em ordem crescente. Vamos denotar a
menor observação por $x_{(1)},$ a segunda por $x_{(2)},$ e assim por diante, obtendo-se
\begin{equation}\label{est_ordem}
x_{(1)}\leq x_{(2)}\leq \cdots \leq x_{(n-1)}\leq x_{(n)}.
\end{equation}
As observações ordenadas como em (\ref{est_ordem}) são chamadas {\bf estatísticas de ordem}. Com esta notação, a mediana da variável $X$ pode ser definida como:
\[
md(X)=\left\{
\setlength\arraycolsep{0pt}
\begin{array}{lr}
X_{(\frac{n+1}{2})},                                             &\quad \textrm{se}\quad $n$\quad \textrm{é impar,}\\
&\\
\frac{X_{(\frac{n}{2})}+X_{(\frac{n}{2}+1)}}{2},  &\quad \textrm{se}\quad $n$\quad \textrm{é par.}\\
\end{array}
\right.
\]
\end{block}
\end{frame}

\begin{frame}{}
\frametitle{}
\begin{block}{}
\justifying
A mediana é uma medida mais robusta que a média, quando submetida a mudanças nos valores observados, ou a incorporação de mais observações no conjunto de dados original.
\end{block}
\end{frame}

\begin{frame}{}
\frametitle{}
\begin{block}{}
\justifying
\begin{figure}[h]
\centering
\caption{Visualização Geométrica da moda, média e mediana de uma função densidade de probabilidade arbitrária}
\begin{tikzpicture}[scale=1]
\draw[line width=2, fill=green] (-4.9,1.7) .. controls (-4.2,2.4) and (-4.3,3.9) .. (-3.7,3.9) .. controls (-3,3.9) and (-2.5,2.3) .. (1.1,1.7) .. controls (-1,1.7) and (-2.1,1.7) .. (-4.9,1.7);
\draw[line width=2] (-4.7,3.95) -- (-2.4,3.95);
\draw[line width=2] (-3.7,3.95) -- (-3.7,1.2);
\node at (-0.9,3.4) {\bf{Moda}};

\draw[line width=2, fill=orange] (-4.9,-1.5) .. controls (-4.2,-0.8) and (-4.3,0.7) .. (-3.7,0.7) .. controls (-3,0.7) and (-2.5,-0.9) .. (1.1,-1.5) .. controls (-1,-1.5) and (-2.1,-1.5) .. (-4.9,-1.5);
%\draw[line width=2] (-4.7,3.95) -- (-2.4,3.95);
\draw[line width=2] (-2.8,0.5) -- (-2.8,-1.8);
\node at (-3.5,-0.9) {\Large \bf{$50\%$}};
\node at (-1.9,-0.9) {\Large \bf{$50\%$}};
\node at (-0.9,0) {\bf{Mediana}};
\draw[line width=2, fill=red] (2,0) .. controls (2.7,0.7) and (2.6,2.2) .. (3.2,2.2) .. controls (3.7,2.2) and (4.4,0.5) .. (8,0) .. controls (5.9,0) and (4.8,0) .. (2,0);
%\draw[line width=2] (-4.7,3.95) -- (-2.4,3.95);
\draw[line width=2] (-3.7,3.95) -- (-3.7,1.2);
\draw[line width=2, fill=gray] (3.8,0) -- (3.3,-0.6) -- (4.3,-0.6) -- (3.8,0);
\draw[line width=2] (3,-0.6) -- (4.6,-0.6);
\draw[line width=1] (3.2,-0.6) -- (3,-0.8);
\draw[line width=1] (3.4,-0.6) -- (3.2,-0.8);
\draw[line width=1] (3.6,-0.6) -- (3.4,-0.8);
\draw[line width=1] (3.8,-0.6) -- (3.6,-0.8);
\draw[line width=1] (4,-0.6) -- (3.8,-0.8);
\draw[line width=1] (4.2,-0.6) -- (4,-0.8);
\draw[line width=1] (4.4,-0.6) -- (4.2,-0.8);
\draw[line width=1] (4.6,-0.6) -- (4.4,-0.8);
\draw (1.5,0.2) .. controls (1.3,0) and (1.3,-0.2) .. (1.5,-0.5);
\draw (1.3,0.2) .. controls (1.1,0) and (1.1,-0.2) .. (1.3,-0.5);
\draw (8.2,0.4) .. controls (8.4,0.2) and (8.4,-0.2) .. (8.3,-0.4);
\draw (8.5,0.4) .. controls (8.7,0.2) and (8.7,-0.2) .. (8.6,-0.4);
\node at (4.8,2) {\bf{Média}};
\end{tikzpicture}
\end{figure}
\end{block}
\end{frame}

\begin{frame}%[allowframebreaks]
\frametitle{\bf Referências}
\bibliography{Referencias.bib}
\end{frame}


\end{document}
